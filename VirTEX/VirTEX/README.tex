%
% README.tex
%
% $Id: README.tex,v 1.1 2011/07/09 01:45:30 jcp Exp jcp $

\documentclass[12pt, letterpaper, oneside]{article}

% Load any additional packages
\usepackage[cm]{fullpage}

% Document specific tweaks
\pagestyle{empty}

% Begin the document
\begin{document}

% Document information
\title{Virtual Timeseries EXperiment (VirTEX) - Quick Start}
\author{John C.~Peterson\\Michael B.~Porter}
\date{8 July, 2011}
% Document title page
\maketitle
\thispagestyle{empty}

% Introduction
\section{Introduction}
\parindent 0pt
\parskip 12pt

The {\em VirTEX} family of algorithms (short for Virtual Timeseries
EXperiment) were designed to model the propagation through the underwater
sound channel of a known timeseries transmitted from a hypothetical
source. The algorithm computes the timeseries that would be observed at
a hypothetical receiver, taking into account the effects of multipath
and doppler introduced by environmental motion.  The algorithms operate
by post processing the outputs produced by the BELLHOP ray tracing code.

It should be noted that the {\em VirTEX} family of algorithms do
not attempt to model every conceivable aspect of the underwater
sound channel. There are many effects (e.g. volume scattering)
that the models do not attempt to address.

These algorithms were designed primarily for scientists and
engineers interested in the modeling of the propagation of underwater
acoustic communications. (However, they could certainly be useful to
researchers with other related interests).

This document is designed to assist users in getting started with
the use of these algorithms. The history, capabilities and limitations
of the different members of the {\em VirTEX} family of algorithms are
briefly described in section two. Section three presents an overview of
the pre-requisites and the remaining sections present outlines of
running the algorithms (all written in the MATLAB interpreted language).

% Variations
\section{Variations of the VirTEX Algorithm}

This section briefly describes the capabilities and limitations of the
different members of the {\em VirTEX} family of algorithms. A tutorial of
their use in modeling practical problems will be presented later.

\subsection{Full VirTEX}
The {\em VirTEX} algorithm was first documented in the journal paper;
{\em Modeling broadband ocean acoustic transmissions with time-varying
sea surfaces}, M.~Siderius, M.~Porter, Journal of the Acoustical Society
of America, vol.~124, no.~1, pp.~137-150. New variants of this algorithm
have since been developed, so it is also referred to as {\em Full VirTEX}
in this document to avoid confusion.

{\em Full VirTEX} is capable of handling almost any type of steady and
unsteady environmental motion. One notable exception is the modeling
of breaking waves, (this limitation is really because the BELLHOP ray
tracing code assumes that the sea surface heights can be described by
a single valued function of position and time, an assumption that is
violated in the case of breaking waves).

The basic idea underlying {\em Full VirTEX} is to approximate the
motion of the environment itself by a sequence of snapshots or freeze
frames. A BELLHOP ray tracing computation is performed for {\em each}
snapshot or frame, and the results are assembled to obtain a piecewise
constant representation of the generalized arrival function. For practical
problems with typical environmental motion, large numbers of frames
are required (many thousands is not an uncommon number), making it by
far the most demanding of computational resources of the {\em VirTEX}
family of algorithms.

\subsection{VirTEX for Platform Motion}
{\em VirTEX for Platform Motion} was designed to address {\em steady
motion of the source, and/or the receiver}.  For each eigenray, the
steady motion manifests itself as a uniform resampling (or stretching)
of the known waveform transmitted by the source.  The resampling can
be accomplished very efficiently by pre-computing a ``library'' of
pre-stretched waveforms using the Chirp Z transform and then performing a
look-up for each eigenray. As a result, {\em VirTEX for Platform Motion}
requires relatively modest computational resources.

\subsection{VirTEX for Sea Surface Motion}
{\em VirTEX for Sea Surface Motion} is capable of addressing steady
motion of the source, and/or the receiver, as well as unsteady sea surface
motion.  The algorithm addresses the doppler introduced by unsteady
sea surface motion by reading the eigenray data produced by BELLHOP. At
each interaction of an eigenray with the moving sea surface, first order
adjustments are made to the arrival times to account for the change in
the path length from the source to receiver The computational resource
requirements of {\em VirTEX for Sea Surface Motion} are greater than
those of {\em VirTEX for Platform Motion}, but still significantly less
than those of {\em Full VirTEX}.

\section{Pre-Requisites}

The requisite model inputs and other software packages for running
{\em VirTEX} are briefly described in this section. A more detailed
discussion of pre-requisites specific to the different versions of
{\em VirTEX} will be presented in the tutorial sections that follow.

\begin{description}
\item[MATLAB software]\hfill \\
All of the variants of the {\em VirTEX} algorithm were written
in MATLAB, so a licensed copy of that software is required. The OCTAVE
interpreter, available at no cost under the GNU General Public License
may also work (but has not been tested by the authors as of this writing).

\item[The BELLHOP ray tracing program.]\hfill \\
As noted above, all of the variants of the {\em VirTEX} algorithm
operate by post-processing the output of the BELLHOP ray tracing
program. While the underlying algorithm is compatible with any
ray tracing software, this implementation currently supports {\em only}
outputs produced by BELLHOP. The latest copy of BELLHOP is available
free of charge under the terms of GNU General Public License from
the Ocean Acoustics library hosted by HLS Research, Inc. See {\tt
http://oalib.hlsresearch.com/Rays/index.html}

\item[Software for modeling the sea surface height.]\hfill \\
While the {\em Full VirTEX} and the {\em VirTEX for Sea Surface Motion}
algorithms were designed to model the effects of unsteady sea surface
motion, they do {\em not} include any specific capability for modeling of
the sea surface motion itself.  An optional resource that the authors have
found to be helpful in this regard is the WAFO (Wave Analysis for Fatigue
and Oceanography) toolbox that is also written largely in MATLAB. It can
be downloaded from {\tt http://www.maths.lth.se/matstat/wafo/}. (When
installing WAFO, it suffices to just unpack the files, you {\em do not}
need to build the MEX objects). For those users that have their own
software for modeling sea surface heights, the WAFO toolbox is not
required.

\item[A description of environment of interest.]\hfill \\
The most important ingredient for successful acoustic
propagation modeling is access to environmental information for the
site of interest. These include, but are not limited to bathymmetry,
sound speed profile data, and bottom scattering properties. These items
are not needed by {\em VirTEX} itself, but are a pre-requisite for
the preparation of the input files for running the
BELLHOP ray tracing program.

\item[A digitally sampled copy of the transmitted timeseries.]\hfill \\
The timeseries transmitted by the source must be available in
digital form, sampled at a rate high enough to preclude aliasing
effects.  This is generally not a problem in the case of software
based acoustic modems, since this is exactly what they output. For
hardware acoustic modems, one must capture the transmitted timeseries
using a general purpose computer equipped with an analog to digital
converter and appropriate software.
\end{description}

\section{VirTEX for Platform Motion}
This section provides a brief tutorial on the use of the {\em VirTEX for
Platform Motion} software. For users that have no previous experience
with any of the {\em VirTEX} family of algorithms, this is a good place
to start due to it's relative simplicity. Here are the basic steps
involved;

\begin{description}
\item {\bf Generate a BELLHOP binary arrivals file for your environment.}\hfill \\
The first step in the process is to prepare an input file for the BELLHOP
ray tracing program based on the information available for the environment
of interest. As with any propagation modeling effort, first running BELLHOP
in transmission loss and ray modes can provide valuable feedback to the
user as to the validity of the inputs. Once you are satisfied with your
BELLHOP input file, modify it to generate a (binary format) arrivals
file, and rerun BELLHOP. Note that multiple receivers located on a
regularly spaced grid in range, depth can be modeled if so desired.

\item {\bf Generate the ``library'' of pre-stretched source waveforms.}\hfill \\
The next step in the process is to generate a database or library of
pre-stretched versions of the known waveform transmitted by the source
by calling the function {\tt \bf delay\_sum\_lib()}.  The source waveform
must have a little bit of zero padding both before and after the
non-zero envelope of the waveform.  This is to insure that the
doppler stretched waveforms do not get truncated for the case of
receding receivers (where the total duration of the non-zero envelope
will increase). The sample rate of the received waveform computed in
the next step will match that of the source waveform here, so choose
accordingly. The library output by this step is independent of the given
environment, so it can be saved (e.g. as a MATLAB save set), and re-used
later on without having to recompute it.

\item {\bf Compute the received timeseries by calling the {\tt delay\_sum()} function.}\hfill \\
The final step in the process is to call the {\tt \bf delay\_sum()}
function. The primary inputs are the name of the BELLHOP (binary format)
arrivals file, the library of pre-stretched source waveforms, and specific
values of the source and receiver velocity. The function outputs the
timeseries observed at each of the hypothetical receivers defined in
the BELLHOP computation.

\end{description}

\subsection{VirTEX for Platform Motion - MATLAB Functions}
In this section, the specific MATLAB functions that implement the
{\em VirTEX for Platform Motion} algorithm are briefly discussed. A
detailed description of the input and output arguments of these
functions can be obtained from the MATLAB built-in help system.
For example; ``{\tt \bf help delay\_sum\_lib}'' will display
a detailed description of the usage of that function.

\begin{description}

\item {\tt \bf delay\_sum\_lib.m}\hfill \\
The {\tt \bf delay\_sum\_lib()} function creates a ``library''
of pre-stretched or ``dopplerized'' versions of the known waveform
transmitted by the source. The caller specifies the minimum and maximum
values of the expected receiver velocity, and the number of velocity
bins. The output is a self-contained MATLAB structure that can be passed
to the {\tt \bf delay\_sum()} function. This function internally calls
the {\tt \bf arbitrary\_alpha()} function, which is located in the {\tt
Matlab/Misc} sub-directory of the Acoustics Toolbox distribution.

\item {\tt \bf delay\_sum.m}\hfill \\
The {\tt \bf delay\_sum()} function is the core function of {\em VirTEX for
Platform Motion}. It computes the convolution of the source timeseries
with the channel impulse response computed by BELLHOP. The primary inputs
are the name of the BELLHOP arrivals file, the structure containing the
library of pre-stretched source timeseries, and velocity vectors for both
the source and receiver.

\end{description}

\subsection{VirTEX for Platform Motion - Example Scripts}
In this section, some MATLAB scripts are identified that present some
simple examples of how to use the {\em VirTEX for Platform Motion}
functions to solve a practical problem. These scripts are located in
the {\tt \bf Examples} sub-directory.

\begin{description}

\item {\tt \bf Example\_MP.m}\hfill \\
The {\tt \bf Example\_MP} script presents an example of a fixed source
and a grid of receivers, each moving with a specified velocity. For
efficiency reasons, it is preferable to compute the ``library'' of
pre-stretched or ``dopplerized'' versions of the source waveform once,
and saving the result as a MATLAB saveset. (The example script recomputes
the library each time it is run to make it self-contained). You must run
BELLHOP with the {\tt \bf Example\_MP.env} input file {\em before}
running this example.

\end{description}

\section{VirTEX for Sea Surface Motion}
This section provides a brief tutorial on the use of the {\em VirTEX for
Sea Surface Motion} software. Here are the basic steps involved;

\begin{description}
\item {\bf Generate a BELLHOP input file for your environment.}\hfill \\
The first step in the process is to prepare an input file for the BELLHOP
ray tracing program based on the information available for the environment
of interest. At the time of this writing, a single invocation of {\em
VirTEX for Sea Surface Motion} can address only {\em one} receiver.
If you wish to model multiple receivers, you must address them one at
a time. (The MATLAB utility functions for reading and writing BELLHOP
input files can be used to automate that process).

\item {\bf Generate BELLHOP arrivals and eigenray files.}\hfill \\
You will need to run BELLHOP twice, once in arrivals mode, and then again
in eigenray mode (with everything else being identical for the two runs).

\item {\bf Upsample the source timeseries and compute the Hilbert transform.}\hfill \\
The {\em VirTEX for Sea Surface Motion} algorithm computes the receiver
timeseries associated with each eigenray by computing the time that each
received sample was transmitted by the source. This computed time is then
rounded to the nearest sample of an upsampled copy of the source
timeseries (computed by calling the {\tt \bf fourier\_upsample()}
function and then computing the Hilbert transform of the result).
The error introduced by this interpolation can be reduced by
upsampling to higher sample rates (at the expensive of additional
computational resources).

\item {\bf Compute the received timeseries by calling the {\tt \bf delay\_sum\_surface()} function.}\hfill \\
The final step in the process is to call the {\tt \bf
delay\_sum\_surface()>} function. The primary inputs are the name of
the BELLHOP (binary format) arrivals file, the name of the BELLHOP
eigenray file, the Hilbert transform of upsampled source timeseries,
an array of surface heights or handle of a function that can compute
surface heights, and velocity vectors for the source and receiver.
The function outputs the timeseries observed at the single receiver
defined in the BELLHOP computation.

\end{description}

\subsection{VirTEX for Sea Surface Motion - MATLAB Functions}
In this section, the specific MATLAB functions that implement the
{\em VirTEX for Sea Surface Motion} algorithm are briefly discussed. A
detailed description of the input and output arguments of these
functions can be obtained from the MATLAB built-in help system.
For example; ``{\tt \bf help delay\_sum\_surface}'' will display
a detailed description of the usage of that function.

\begin{description}
\item {\tt \bf delay\_sum\_surface.m}\hfill \\
The {\tt \bf delay\_sum\_surface()} function is the core function of
{\em VirTEX for Sea Surface Motion}. The caller must pass the
names of the BELLHOP arrival and eigenray files, the Hilbert transform
of the transmitted waveform (usually a suitably upsampled copy), values for
the source and receiver velocity, and either a matrix of surface heights over
appropriate ranges and times or the handle of a function that can be called
to compute the height for a given range and time on demand. 

\item {\tt \bf fourier\_upsample.m}\hfill \\
The {\tt \bf fourier\_upsample} function resamples a timeseries
to a new sample rate that is a prescribed integer multiple of the original
sample rate. It accomplishes this by zero padding in the frequency domain,
so the Fourier content is not altered. The error introduced by interpolation
in time performed by  {\tt \bf delay\_sum\_surface} can be reduced by
passing the transmit timeseries at a sample rate that is at least a
few integer multiples of the desired sample rate of the computed
receive timeseries.

\item {\tt \bf read\_arrivals\_bin.m}\hfill \\
The {\tt \bf read\_arrivals\_bin()} function is called internally by the
{\tt \bf delay\_sum\_surface()} function to read the binary arrivals file
produced by BELLHOP. For the typical user, direct calls to this function
are not required.

\item {\tt \bf read\_rayfil.m}\hfill \\
The {\tt \bf read\_rayfil()} function is called internally by the
{\tt \bf delay\_sum\_surface()} function to read the eigenray data file
produced by BELLHOP. For the typical user, direct calls to this function
are not required.

\end{description}

\subsection{VirTEX for Sea Surface Motion - Example Scripts}
In this section, some MATLAB scripts are identified that present some
simple examples of how to use the {\em VirTEX for Sea Surface Motion}
functions to solve a practical problem. These scripts are located in
the {\tt \bf Examples} sub-directory.

\begin{description}

\item {\tt \bf Example\_Lite\_CS.m}\hfill \\
The {\tt \bf Example\_Lite\_CS} script presents an example of a swell
wave that is moving perpendicular (cross swell) to the vertical plane
containing the source and receiver. The surface heights are communicated
to the {\tt \bf delay\_sum\_surface()} function by passing it the handle
to the {\tt \bf gravity\_wave} function that computes the surface wave
height on demand for given range and time values. You must run BELLHOP
with the {\tt \bf Example\_Lite.env} input file {\em before} running
this example, once in arrivals mode, and again in eigenray mode.

\item {\tt \bf Example\_Lite\_DS.m}\hfill \\
The {\tt \bf Example\_Lite\_DS} script presents an example of a
swell wave that is moving parallel (down swell) to the vertical plane
containing the source and receiver. The surface heights are passed
to the {\tt \bf delay\_sum\_surface()} function as a matrix of computed
values over appropriate range and time values. This example utilizes
the WAFO toolbox, which must be installed on your system before you can
run this example.  Once installed, update your MATLAB paths by running
{\tt \bf initwafo('minimal')} located in the top level directory of the
WAFO toolbox. You must run BELLHOP with the {\tt \bf Example\_Lite.env}
input file {\em before} running this example, once in arrivals mode,
and again in eigenray mode.

\item {\tt \bf gravity\_wave.m}\hfill \\
The {\tt \bf gravity\_wave()} function computes the height of a simple
(gravity) swell wave for a given range and time. It is called by the
{\tt \bf Example\_Lite\_CS.m} example script described above.

\item {\tt \bf add\_noise.m}\hfill \\
The {\tt \bf add\_noise} script assembles a collection of MATLAB save
sets containing simulated receive timeseries, then adds band limited
Gaussian white noise. It writes out a single "playback" file (designed
for playback to a hardware modem) in signed 16 bit integer format.

\end{description}

\section{Full VirTEX}
This section provides a brief tutorial on the use of the {\em Full VirTEX}
software. It should be noted that while the underlying concepts behind
the {\em Full VirTEX} algorithm are not that much more complicated than
the other variants of the algorithm, the software implementation of the
algorithm is {\em significantly} more complex. In particular, much of
the current code is operating system dependent, and only runs on UNIX
systems as of this writing.


\subsection{Full VirTEX - MATLAB Functions}
In this section, the specific MATLAB functions that implement the {\em
Full VirTEX} algorithm are briefly discussed. A detailed description
of the input and output arguments of these functions can be obtained
from the MATLAB built-in help system.  For example; ``{\tt \bf help
moving\_environment}'' will display a detailed description of the usage
of that function.

\begin{description}
\item {\tt \bf Bellhop\_execute.m}\hfill \\
The {\tt \bf Bellhop\_execute()} function is called by the {\tt \bf
moving\_environment()} function to manage the execution of BELLHOP for {\em
each} time snapshot or frame (one for each time value in the array of
surface heights). This function contains code that is operating system
dependent and currently works only on UNIX systems. For the typical user,
direct calls to this function are not required.

\item {\tt \bf Bellhop\_inputs.m}\hfill \\
The {\tt \bf Bellhop\_execute()} function is called by the {\tt \bf
moving\_environment()} function to construct BELLHOP input files for {\em
each} time snapshot or frame (one for each time value in the array of
surface heights).  This function contains code that is operating system
dependent and currently works only on UNIX systems. For the typical user,
direct calls to this function are not required.

\item {\tt \bf load\_arrfils.m}\hfill \\
The {\tt \bf load\_arrfils()} function is called by the {\tt \bf
moving\_environment()} function to read all of the BELLHOP arrival files
generated by the call to the {\tt \bf Bellhop\_execute()} function. It
collates all of the arrival information and returns it as a MATLAB struct.
For the typical user, direct calls to this function are not required.

\item {\tt \bf moving\_environment.m}\hfill \\
The {\tt \bf moving\_environment()} function is the core function of
{\em Full VirTEX}. Unlike the other variants of the algorithm, {\em Full VirTEX}
handles the execution of BELLHOP by calling the {\tt \bf Bellhop\_inputs()}
and {\tt \bf Bellhop\_execute()} functions.

\end{description}

\subsection{Full VirTEX - Example Scripts}
In this section, some MATLAB scripts are identified that present some
simple examples of how to use the {\em Full VirTEX} functions to solve
a practical problem. These scripts are located in the {\tt \bf Examples}
sub-directory.

\begin{description}
\item {\tt \bf Example\_Full\_DS.m}\hfill \\
The {\tt \bf Example\_Full\_DS} script presents an example of a swell
wave that is moving parallel (down swell) to the vertical plane containing
the source and receiver. You must edit this script and modify the line
where the variable {\tt \bf bparams.bellhop\_path} is assigned a value
(which specifies where BELLHOP is installed on your system).

\end{description}

\end{document}
